\documentclass[10pt,oneside]{article}
\usepackage{geometry}
\usepackage[T1]{fontenc}
\usepackage{enumitem}

\pagestyle{empty}
\geometry{letterpaper,tmargin=.63in,bmargin=.5in,lmargin=.5in,rmargin=.5in,headheight=0in,headsep=0in,footskip=.3in}

\setlength{\parindent}{0in}
\setlength{\parskip}{0in}
\setlength{\itemsep}{0in}
\setlength{\topsep}{0in}
\setlength{\tabcolsep}{0in}

% Name and contact information
\newcommand{\name}{Kyle Brieden}
\newcommand{\addr}{Atlanta, Georgia}
\newcommand{\phone}{(979) 575-9459}
\newcommand{\email}{kyle@brieden.org}
\newcommand{\github}{github.com/txmoose}


%%%%%%%%%%%%%%%%%%%%%%%%%%%%%%%%%%%%%%%%%%%%%%%%%%%%%%%%%
% New commands and environments

% This defines how the name looks
\newcommand{\bigname}[1]{
    \begin{center}\fontfamily{phv}\selectfont\Huge\scshape#1\end{center}
}

% A ressection is a main section (<H1>Section</H1>)
\newenvironment{ressection}[1]{
    \vspace{4pt}
    {\fontfamily{phv}\selectfont\Large#1}
    \begin{itemize}
    \vspace{3pt}
}{
    \end{itemize}
}

% A resitem is a simple list element in a ressection (first level)
\newcommand{\resitem}[1]{
    \vspace{-4pt}
    \item \begin{flushleft} #1 \end{flushleft}
}

% a resobj is an objective item that shouldn't have a bullet point
\newcommand{\resobj}[1]{
    \vspace{-4pt}
    \item[\null]  \begin{flushleft} #1 \end{flushleft}
}

% A ressubitem is a simple list element in anything but a ressection (second level)
\newcommand{\ressubitem}[1]{
    \vspace{-1pt}
    \item \begin{flushleft} #1 \end{flushleft}
}

% A resbigitem is a complex list element for stuff like jobs and education:
%  Arg 1: Name of company or university
%  Arg 2: Location
%  Arg 3: Title and/or date range
\newcommand{\resbigitem}[3]{
    \vspace{-5pt}
    \item
    \textbf{#3} \\
    \textit{#1}---#2
}

% This is a list that comes with a resbigitem
\newenvironment{ressubsec}[3]{
    \resbigitem{#1}{#2}{#3}
    \vspace{-2pt}
    \begin{itemize}
}{
    \end{itemize}
}

\newenvironment{resedusec}[3]{
    \resbigitem{#1}{#2}{#3}
    \vspace{-2pt}
}


% This is a simple sublist
\newenvironment{reslist}[1]{
    \resitem{#1}
    \vspace{-5pt}
    \begin{itemize}
}{
    \end{itemize}
}

%%%%%%%%%%%%%%%%%%%%%%%%%%%%%%%%%%%%%%%%%%%%%%%%%%%%%%%%%
% Now for the actual document:

\begin{document}

\fontfamily{lmss} \selectfont

% Name with horizontal rule
\bigname{\name}

\vspace{-8pt} \rule{\textwidth}{1pt}

\vspace{-1pt} {\small\itshape \addr \hfill \github; \email; \phone}

\vspace{8 pt}

%%%%%%%%%%%%%%%%%%%%%%%%
\begin{ressection}{Professional Summary}

    \begin{resobj}{Experienced Staff Systems Engineer with a demonstrated history in DevOps technologies such as Kubernetes, Docker, Ansible, GitHub Actions, and Terraform.  Skilled in modernizing application deployment strategies and strongly believe in the "cattle not pets" school of thought with infrastructure.  Seeking to grow skills in production engineering and help bring companies into modern production practices.}
   \end{resobj}

\end{ressection}
%%%%%%%%%%%%%%%%%%%%%%%%
\begin{ressection}{Experience}
    \begin{ressubsec}{The Home Depot}{Atlanta, GA}{Staff Systems Engineer: June 2021-Current}
        \ressubitem{Built out new Kubernetes-based on-prem production app environment for THD App Teams which includes out of the box metrics and logging, alerting, and integration with other THD Enterprise services such as Artifactory to allow app teams to focus on their apps rather than supporting functions}
        \ressubitem{Infrastructure to support the on-prem environment is deployed into various GCP services}
        \ressubitem{Assisted in developing a custom Terraform provider and custom code in GoLang, Python, and Bash in support of the new on-prem platform}
        \ressubitem{Supported sunsetting of legacy platforms such as Pivotal Cloud Foundry foundations and Hadoop clusters}
        \ressubitem{Day to day team leadership roles such as backlog grooming, sprint planning, quarterly team objective setting, and team building activities}
        \ressubitem{On-Call rotation to respond to business-impacting events as quckly as possible}
        \ressubitem{White-glove customer service to support THD app teams on our platforms}
    \end{ressubsec}

    \begin{ressubsec}{Riot Games}{St. Louis, MO}{Senior Systems Engineer: December 2017-May 2021}
        \ressubitem{Replaced manual build pipeline with Jenkins and AWS-based CI/CD pipelines for the game Legends of Runeterra (LoR)}
        \ressubitem{Support day to day operations of LoR as part of the Production Engineering group, as well as automate tasks where able}
        \ressubitem{Discovered and removed just over \$400,000/yr of unnecessary AWS spend}
        \ressubitem{Previous roles at Riot: Infrastructure Platform - Datacenter, Player Platform - Accounts}
    \end{ressubsec}

    \begin{ressubsec}{Macy's Systems \& Technology}{Johns Creek, GA}{Analyst 2 - Linux Automation: September 2016-December 2017}
        \ressubitem{Developed automation to deliver prototyping environments for internal customers more quickly via Puppet and vSphere}
        \ressubitem{Provide day to day operational support for Macy's infrastructure at large}
        \ressubitem{Linux Automation}
    \end{ressubsec}

    \begin{ressubsec}{Kickboard For Schools}{New Orleans, LA}{DevOps Engineer: June 2016-September 2016}
        \ressubitem{Helped develop foundation for growth of platform from startup to established product}
        \ressubitem{Provide team with Systems Engineering experience and knowledge}
        \ressubitem{DevOps Engineer - Remote - Contract}
    \end{ressubsec}

    \begin{ressubsec}{WP Engine}{Austin, TX}{Operations Engineer: June 2014-April 2016}
        \ressubitem{Responsible for farm-wide monitoring and first response to assist Systems Engineering team}
        \ressubitem{Production farm maintenance, deploying updates, and assisting support staff when needed}
        \ressubitem{Operations Center - Third Shift}
    \end{ressubsec}

    \pagebreak

    \begin{ressubsec}{Rackspace}{San Antonio, TX}{Linux Administrator II: December 2012-June 2014}
        \ressubitem{Linux Subject Matter Expert (SME) for Rackspace's most strategic accounts, including Electronic Arts}
        \ressubitem{Provided chat and phone support, as well as ticket work for SMB customers from \$15 per month up to \$10,000 per month in prior role}
        \ressubitem{Previous role at Rackspace: Linux Administrator I - Small to Medium Businesses (SMB) Support}
        \ressubitem{Critial Accounts - Enterprise Support}
    \end{ressubsec}

\end{ressection}

\begin{ressection}{Related Experience}
    \begin{reslist}{I stood up an LLC to cover some side projects helping friends with small businesses. These projects include:}
        \resitem{CI/CD via GitHub Actions, AWS ECR, AWS EKS, and FluxCD}
        \resitem{Monitoring, log aggregation, dashboarding, and alerting through Prometheus, Loki, and Grafana}
    \end{reslist}
    \begin{reslist}{I maintain a small home lab that includes:}
        \resitem{Raspberry Pi-based Kubernetes cluster}
        \resitem{Proxmox hypervisor}
        \resitem{PiHole DNS Ad Blocking}
        \resitem{Wireguard VPN server}
        \resitem{Various services for personal use and programming projects}
    \end{reslist}
    \resitem{I wrote a small web app to replace paper bid sheets with QR codes for a silent auction for the nonprofit where my wife works.}
    \resitem{I quickly moved from L3 support to start the Systems Operations Center (SOC) team to provide 3rd shift Systems Operations and Engineering coverage. This team quickly grew from only myself to 4 members providing 7-night per week coverage while the company was in its rapid growth phase.}
    \resitem{At Rackspace, I was a member of a community support effort engaging customers via community forums in addition to my front line support role.}
    \resitem{Amateur Radio Operator - KF5MWZ, General Class}
    \resitem{I wrote this resume in \LaTeX.}
    
\end{ressection}

\begin{ressection}{Education}
    \begin{resedusec}{Texas A\&M University} {College Station, TX} {Telecommunications Engineering Technology}
    \end{resedusec}
\end{ressection}
\end{document}

